\documentclass{article}
\usepackage{graphicx} % Required for inserting images
\usepackage{geometry} 
\usepackage{amsmath}


\geometry{headsep=10pt,top=20pt}
% \DeclareMathSizes{12}{20}{14}{10}

\title{LaTeX Math 2}
\author{Kiratu Onesmas}
\date{August 2023}

\begin{document}

\maketitle

\section{Equations}
\LaTeX\vspace{10pt}


\subsection{Adding Text to Equations}

% with referencing equations
\begin{equation}
50 \, apples \times \, 100 \, apples = \, lots \, of \, apples
\end{equation}
\vspace{10pt}

% without referencing equations

\begin{equation*}
50 \text{ apples} \times 100 \text{ apples} = \text{lots of apples}
\end{equation*}
\vspace{10pt}

\subsection{Using mBox to Space Text}
\begin{equation}
50 \mbox{ apples} \times 100 \mbox{ apples} =
\mbox{lots of apples}
\end{equation}

\subsection{Formatted Text}
% use \textrm, \textit, textbf etc instead of mbox
%textrm = Roman Text

\begin{equation}
50 \textrm{apples} \times 100 \textbf{apples} =
\textit{lots of apples}
\end{equation}
\vspace{10pt}

\begin{equation}
50 \textit{ apples} \times 100 \texttt{ apples} =
\textbf{ lots of apples}
\end{equation}
\vspace{10pt}


% Math Formatting Commands 
% \mathrm{...} Roman
% \mathit{...} Italic
% \mathbf{...} Bold
% \mathsf{...} Sans serif
% \mathtt{...} Typewriter
% \mathcal{...}

In mathematics, the speed of light is denoted by $c = 3 \times 10^8 \, \mathrm{m/s}$.

\subsection{Changing Text Size of Equations}
% \displaystyle Size for equations in display mode
% \textstyle Size for equations in text mode
% \scriptstyle Size for first sub/superscripts
% \scriptscriptstyle Size for subsequent sub/superscripts

\begin{equation} x = a_0 + \frac{1}{a_1 + \frac{1}{a_2 + \frac{1}{a_3 + a_4}}}
\end{equation}
\vspace{10pt}

%\DeclareMathSizes{10}{18}{12}{8} % For size 10 text
%\DeclareMathSizes{11}{19}{13}{9} % For size 11 text
%\DeclareMathSizes{12}{20}{14}{10} % For size 12 

\subsection{Multi-lined Equations (eqnarray environment}

\begin{tabular}{ r l }
\(10xy^2+15x^2y-5xy\) & \(= 5\left(2xy^2+3x^2y-xy\right)\)\\
& \(= 5x\left(2y^2+3xy-y\right)\) \\
& \(= 5xy\left(2y+3x-1\right)\)
\end{tabular}
\vspace{10pt}

\begin{tabular}{ r c l }
\(10xy^2+15x^2y-5xy\) & \(=\) & \(5\left(2xy^2+3x^2y-xy\right)\) \\
& \(=\) & \(5x\left(2y^2+3xy-y\right)\) \\
& \(=\) & \(5xy\left(2y+3x-1\right)\)
\end{tabular}
\vspace{10pt}

\begin{eqnarray}
10xy^2+15x^2y-5xy & = & 5\left(2xy^2+3x^2y-xy\right) \nonumber \\
& = & 5x\left(2y^2+3xy-y\right) \nonumber \\
& = & 5xy\left(2y+3x-1\right)
\end{eqnarray}
\vspace{10pt}

\subsection{Breaking Up Long Equations}
\begin{eqnarray*}
\left(1+x\right)^n & = & 1 + nx + \frac{n\left(n-1\right)}{2!}x^2 \\
& & {} + \frac{n\left(n-1\right)\left(n-2\right)}{3!}x^3 \\
& & {} + \frac{n\left(n-1\right)\left(n-2\right)\left(n-3\right)}{4!}x^4 \\
& & {} + \ldots
\end{eqnarray*}
\vspace{10pt}

\begin{eqnarray*}
\lefteqn{\left(1+x\right)^n = } \\
& & 1 + nx + \frac{n\left(n-1\right)}{2!}x^2 + \\
& & \frac{n\left(n-1\right)\left(n-2\right)}{3!}x^3 + \\
& & \frac{n\left(n-1\right)\left(n-2\right)\left(n-3\right)}{4!}x^4 + \\
& & \ldots
\end{eqnarray*}
\vspace{10pt}

\subsection{Controlling Horizontal Spacing}
\subsubsection{f with large curls} 

\[f(n) = \left\{
\begin{array}{l l}
n/2 & \quad \mbox{if $n$ is even}\\ -(n+1)/2 & \quad \mbox{if $n$ is odd}\\ \end{array} \right. \]

\[
f(n) = \left\{
\begin{array}{l l}
n/2 & \quad \mbox{if $n$ is even}\\ -(n+1)/2 & \quad \mbox{if $n$ is odd}\\ \end{array} \right. 
\]
\vspace{10pt}

\[f(n) =
\begin{cases}
n/2 & \text{if $n$ is even} \\
-(n+1)/2 & \text{if $n$ is odd}
\end{cases}
\]
\vspace{10pt}

\[ \int y \; \mathrm{d}x \]
\vspace{10pt}

\[\left(\!\!\!
\begin{array}{c}
n \\
r
\end{array}
\!\!\!\right) = {^n}C_r = \frac{n!}{r!(n-r)!}
\]
\vspace{10pt}

%with a bigger space
\[\left(
\begin{array}{c}
n \\
r
\end{array}
\right) = {^n}C_r = \frac{n!}{r!(n-r)!}
\]


f^{0}(\lambda) = 
\begin{cases} 
0 & \text{if } \lambda \leq 0 \\
k_{0}(1 + \lambda )e^{-\lambda} & \text{if } \lambda > 0 
\end{cases}

f^{0}(\lambda) = 
\begin{cases} 
0 &  \lambda \leq 0 \\
k_{0}(1 + \lambda )e^{-\lambda} &  \lambda > 0 
\end{cases}


Find the cumulative distribution function for the doctor's prior distribution and hence find the values \theta _1, \theta _2, such that in the biologists prior distribution Pr (\theta < \theta _1) = Pr(\theta > \theta _2)
= 0.05



\end{document}
