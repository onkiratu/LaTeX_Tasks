documentclass{article}
documentclass{standalone}
usepackage{graphicx} % Required for inserting images
usepackage{graphicx} % Required for inserting images
%usepackage[left=0.5in, right=0.5in, top=0.5in, bottom=0.5in]{geometry}
%usepackage[left=1.5cm, right=1cm, top=0.5cm, bottom=1.5cm]{geometry}
usepackage[left=1.5cm, right=1.5cm, top=0.5cm, bottom=1.5cm]{geometry}
usepackage{amsmath}
usepackage{amssymb}
usepackage{amsfonts}
usepackage{amsthm}
usepackage{ulem}
usepackage{bm}
usepackage{tikz}
usepackage{enumitem}

date{}

begin{document}
fontsize{13}{15} selectfont %This is 13pt text with 15pt line spacing.

begin{center}
 text{Potterhouse School. hspace{1cm} Year 6 Math - T2 W1 Homework Task 2.} qquad  
end{center}  

% Name ...........................................................  hspace{0.5cm}  Date ....................... hspace{0.5cm}  Class ......hspace{0.5cm} [20 marks]
begin{center}
vspace{5pt} 
textit{(You must show your working in your squared exercise book.)  }
vspace{5pt}
end{center}

1. Solve (2x - 5y + 3z), where (x = 10), (y = 4), and (z = 2). 

2. Evaluate (7p + 2q - 6r), where (p = 5), (q = 3), and (r = 1). 

3. Find the value of (3m - 2n + 5p), given that (m = 12), (n = 8), and (p = 4). 

4. Solve (4x + 6y - 2z), where (x = 3), (y = 9), and (z = 5). 

5. Evaluate (2a - 4b + 7c), where (a = 6), (b = 2), and (c = 10). 

6. Find the value of (5p + 3q - 2r), given that (p = 7), (q = 4), and (r = 1). 

7. Solve (9x - 2y + 4z), where (x = 8), (y = 5), and (z = 3). 

8. Evaluate (6m + 2n - 3p), where (m = 10), (n = 7), and (p = 2). 

9. Find the value of (4a - 3b + 2c), given that (a = 5), (b = 2), and (c = 9). 

10. Solve (7x + 4y - z), where (x = 6), (y = 3), and (z = 2). 



end{document}
