\documentclass{article}
\usepackage{graphicx} % Required for inserting images
\usepackage{graphicx} % Required for inserting images
%\usepackage[left=0.5in, right=0.5in, top=0.5in, bottom=0.5in]{geometry}
\usepackage[left=1cm, right=1cm, top=0.5cm, bottom=1.5cm]{geometry}
\usepackage{amsmath}
\usepackage{amssymb}
\usepackage{amsfonts}
\usepackage{amsthm}
\usepackage{ulem}
\usepackage{bm}
\usepackage{tikz}

\date{}

\begin{document}
\fontsize{13}{15} \selectfont %This is 13pt text with 15pt line spacing.

\section{Order of Operations -  BODMAS/BEDMAS/PEDMAS/BIDMAS}

The acronyms BODMAS/BEDMAS/PEDMAS/BIDMAS are used to help remember the order of operations in mathematics. 

\vspace{5pt}
\\ 
The order of operations is a set of rules that dictate the sequence in which mathematical operations should be performed. 

\vspace{5pt}
\\ 
Each letter stands for something in the different acronyms:

\begin{itemize}
\item \textbf{BODMAS}: Brackets, Orders (e.g. powers and square roots), Division and Multiplication, Addition and Subtraction
\\

\item \textbf{BEDMAS}: Brackets, Exponents, Division and Multiplication, Addition and Subtraction
\\

\item \textbf{PEDMAS}: Parentheses, Exponents, Division and Multiplication, Addition and Subtraction 
\\

\item \textbf{BIDMAS}: Brackets, Indices, Division and Multiplication, Addition and Subtraction
\end{itemize}

\vspace{5pt}
\\

\subsection{Items in the Order of Operations}

\begin{enumerate}
\item \textbf{Brackets/Parentheses}: Evaluate expressions inside brackets or parentheses first.

\begin{itemize}
\item Example: \( (2 + 3) \times 4 = 5 \times 4 \)
\end{itemize}
\\

\item \textbf{Orders/Exponents/Indices}: Evaluate powers, square roots, and other exponential terms.
\begin{itemize}
\item Example: \( 2^3 = 8 \)
\end{itemize}
\item \textbf{Division and Multiplication}: Perform division and multiplication from left to right.
\begin{itemize}
\item Example: \( 8 \div 4 \times 2 = 2 \times 2 \)
\end{itemize}
\item \textbf{Addition and Subtraction}: Perform addition and subtraction from left to right.
\begin{itemize}
\item Example: \( 5 + 3 - 2 = 6 \)
\end{itemize}
\end{enumerate}

\subsection{Important Note}

\begin{itemize}
\item Division and multiplication are of equal precedence; they are performed from left to right.
\item Similarly, addition and subtraction are of equal precedence and are performed from left to right.
\end{itemize}

\subsection{Practice Questions}

\begin{enumerate}
\item \( (4 + 6) \times 2 \)
\item \( 12 \div 4 + 3 \)
\item \( 5 \times 2 + (3 - 1) \)
\item \( 8 \div (4 + 2) \)
\item \( 9 - 3 \times (2 + 1) \)
\end{enumerate}



\end{document}