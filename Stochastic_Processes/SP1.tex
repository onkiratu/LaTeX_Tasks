\documentclass{article}
\usepackage{graphicx} % Required for inserting images
\usepackage{graphicx} % Required for inserting images
\usepackage[left=1.5cm, right=1.5cm, top=1cm, bottom=1.5cm]{geometry}
\usepackage{amsmath}
\usepackage{amssymb}
\usepackage{amsfonts}
\usepackage{amsthm}
\usepackage{ulem}
\usepackage{bm}
\usepackage{tikz}
\usepackage{enumitem}
\usetikzlibrary{shapes,backgrounds}
\usepackage{textcomp} % for \textdollar command

\date{}

\begin{document}
\fontsize{12}{13} \selectfont %This is 12pt text with 14pt line spacing.
%\setcounter{page}{2}

\begin{pmatrix}
    P = 
\end{pmatrix}  \\


\[
P =
\begin{pmatrix}
\frac{1}{2} & \frac{2}{3} & 0 \\
\frac{1}{2} & \frac{1}{4} & \frac{1}{4} \\
1 & 0 & 0 \\
\end{pmatrix}
\vspace{10pt}
\]
\\

\section(b)
(b) For a probability generating function, \( P_{{S}_{N}} = P_{N} (P_{X}(S)). Show that E_(S_{N}) = E(N).E(X) \)


X_{\substack{i \\ j}
\\ 

Consider a homogeneous discrete time Markov chainnX with state space S = {0,1} and the transition matrix \\

P = \begin{pmatrix}
    1 - \alpha & \alpha \\
    \beta & 1 - \beta \\
 \end{pmatrix}



\is an element 

(c) Let \( N \) and \( X_1, X_2, ... \) be independent and identically distributed count random variables. Set \( S_N = 0 and S_N = X_1 + X_2 + ... + X_n. \) If the common probability generating function of \( X_{i}^{'}s \) is \( P_X \) and the probability generating function of \( N is P_N,\)  then show that the probability generating function of \( S_N, P_{{S}_{N}}(S),\)  is of the form \(  P_{{S}_{N}}(S) = P_N (P_X(S)) \) \\


Further show that \( var(S_{N}) = E(N) var(X_{1}) + var(N)[E(X_{1})]^{2}. \) \\



(a) Random walk on triangle. Consider a homogeneous discrete time Markov Chain with the state space \(  S = (0,1,2) \) and the transition matrix  \\

\[
P =
\begin{pmatrix}
0 & \frac{1}{2} & \frac{1}{2} \\
\frac{1}{2} & 0 & \frac{1}{2} \\
\frac{1}{2} & \frac{1}{2} & 0 \\
\end{pmatrix}
\vspace{10pt}
\]
\\
Find \( P^{n} \) and classify the states of \( X. \) \\


\section{a9}

% \( \hat{X} \)

To find the optimal four-step-ahead and five-steps-ahead forecasts for an AR(1) process X_t = \alpha X_{t-1} + e_t, where  \alpha is a constant and e_t is a white noise error term, we use the properties of the AR(1) process. \\

\section{Bayesian Modelling} 
(b)
Let \(  \theta \) be gamma \( (a,b) \) random variable. Write down the density function and evaluate the mean and variance of \( \theta. \) \\
Please show calculations. \\

(c)
Suppose we have two identical urns. Urn A with 5 red balls and 10 green balls, and urn B with 10 red balls and 5 green balls.  We'll select randomly one of the two urns. Then sample with a replacement, that urn to help determine whether to choose A or B. Let's suppose that our Sample \( X = (x_1, x_2, ..., x_n) \) is of size n, and k of the outcomes \( x_1, x_2, ..., x_n \) are red. Determine the posterior probabilities \( P(A|X) \) and \( P(B|X). \) \\


%Q2
(a) Motor insurance claims can be classified into three mutually exclusive types. S, M and L. The proportions of claims in each type are \( 78 \%, 13 \%, \) and \( 9 \%. \) The distribution of the amounts of individual claims in each type can be modelled by the probability density function: \\

\[ 
f(x) = \frac{3 \theta ^{3}}{x^{3}}; x > \theta \\
\]

where x represents the size of an individual claim, The parameters for the three categories are \(  \theta _{S} = \$  100, \theta_{M} = \$ 1500 \) and \(  \theta _{L} = \$ 3000. \) \\

Given only that the amount of an indivdual claim was \(  \$ 4000,\) find the probability that it belongs to each of the three categories. 

\]
% c
After diagnosing a patient, the physician gave the patient a probability of 0.4 of having a disease. The physician then ordered a clinical laboratory test. A positive laboratory test value had a probability of 0.8 of positively identifying the disease in patients with the disease (true positive) and a probability of 0.1 of positive identification of the disease in subjects without the disease (false positive). From the prior information, (physician's diagnosis) and current patient specific data (laboratory test), what is the posterior probability of the patient having the disease using the Bayesian method. \\

%c
If we are interested in the mean \lambda of a Poisson distribution. Given prior distribution for \lambda with density \\


\( f^{0}(\lambda) = \left \{  \)
\( \begin{array}{l l} \)
\( 0 & \quad \mbox{ \lambda \leq 0}\\ \)
\( k_{0}(1 + \lambda )e^{-\lambda} & \quad \mbox{ \lambda > 0} \\ \)
\( \end{array} \right\) 
\[

f(n) = \left\{
\begin{array}{l l}
n/2 & \quad \mbox{if $n$ is even} \\
-(n+1)/2 & \quad \mbox{if $n$ is odd} \\ 
\end{array} \right. 

\]

\end{document}
